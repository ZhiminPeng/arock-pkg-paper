%
\documentclass[twoside,11pt]{article}

% Any additional packages needed should be included after jmlr2e.
% Note that jmlr2e.sty includes epsfig, amssymb, natbib and graphicx,
% and defines many common macros, such as 'proof' and 'example'.
%
% It also sets the bibliographystyle to plainnat; for more information on
% natbib citation styles, see the natbib documentation, a copy of which
% is archived at http://www.jmlr.org/format/natbib.pdf

\usepackage{jmlr2e}
\usepackage{amsmath}

% Definitions of handy macros can go here

%% macros for commenting
\usepackage{mathtools}
\usepackage[normalem]{ulem} % to use \sout
%\usepackage{algorithm,algorithmic}
%\usepackage{graphicx, subfigure}
\usepackage{subcaption}
%\usepackage{url}
%\usepackage{algorithmic}% http://ctan.org/pkg/algorithms
\usepackage{enumerate}
\usepackage[ruled,vlined]{algorithm2e}
\usepackage{amsfonts}
\usepackage{amssymb}
\usepackage{booktabs}
\usepackage{multirow}
\usepackage{mathrsfs}
\usepackage{hyperref}
\usepackage{listings}


%% a light-weight algorithm environment
\newtheorem{algo}{Algorithm}

%% highlighting and commenting
\newcommand{\outline}[1]{{\color{brown}#1}}
\newcommand{\rev}[1]{{\color{blue}#1}}
\newcommand{\commwy}[1]{{\color{red}(#1 -- Wotao)}} % Wotao Yin's comments
\newcommand{\commyx}[1]{{\color{red}(#1 -- Yangyang)}} % Yangyang Xu's comments
\newcommand{\commzp}[1]{{\color{red}(#1 -- Zhimin)}} % Yangyang Xu's comments
\newcommand{\commtw}[1]{{\color{red}(#1 -- Tianyu)}}
\newcommand{\commrw}[1]{{\color{red}(#1 -- Reviewer)}} % Yangyang Xu's comments
\newcommand{\remove}[1]{{}}
\newcommand{\cut}[1]{}



\RequirePackage[normalem]{ulem} %DIF PREAMBLE
\RequirePackage{color}\definecolor{RED}{rgb}{1,0,0}\definecolor{BLUE}{rgb}{0,0,1} %DIF PREAMBLE
\providecommand{\DIFaddtex}[1]{{\protect\color{blue}\uwave{#1}}} %DIF PREAMBLE
\providecommand{\DIFdeltex}[1]{{\protect\color{red}\sout{#1}}}                      %DIF PREAMBLE
%DIF SAFE PREAMBLE %DIF PREAMBLE
\providecommand{\DIFaddbegin}{} %DIF PREAMBLE
\providecommand{\DIFaddend}{} %DIF PREAMBLE
\providecommand{\DIFdelbegin}{} %DIF PREAMBLE
\providecommand{\DIFdelend}{} %DIF PREAMBLE
%DIF FLOATSAFE PREAMBLE %DIF PREAMBLE
\providecommand{\DIFaddFL}[1]{\DIFadd{#1}} %DIF PREAMBLE
\providecommand{\DIFdelFL}[1]{\DIFdel{#1}} %DIF PREAMBLE
\providecommand{\DIFaddbeginFL}{} %DIF PREAMBLE
\providecommand{\DIFaddendFL}{} %DIF PREAMBLE
\providecommand{\DIFdelbeginFL}{} %DIF PREAMBLE
\providecommand{\DIFdelendFL}{} %DIF PREAMBLE
%DIF END PREAMBLE EXTENSION ADDED BY LATEXDIFF
%DIF PREAMBLE EXTENSION ADDED BY LATEXDIFF
%DIF HYPERREF PREAMBLE %DIF PREAMBLE
\providecommand{\DIFadd}[1]{\texorpdfstring{\DIFaddtex{#1}}{#1}} %DIF PREAMBLE
\providecommand{\DIFdel}[1]{\texorpdfstring{\DIFdeltex{#1}}{}} %DIF PREAMBLE



%% macros for letters

\newcommand{\va}{{\mathbf{a}}}
\newcommand{\vb}{{\mathbf{b}}}
\newcommand{\vc}{{\mathbf{c}}}
\newcommand{\vd}{{\mathbf{d}}}
\newcommand{\ve}{{\mathbf{e}}}
\newcommand{\vf}{{\mathbf{f}}}
\newcommand{\vg}{{\mathbf{g}}}
\newcommand{\vh}{{\mathbf{h}}}
\newcommand{\vi}{{\mathbf{i}}}
\newcommand{\vj}{{\mathbf{j}}}
\newcommand{\vk}{{\mathbf{k}}}
\newcommand{\vl}{{\mathbf{l}}}
\newcommand{\vm}{{\mathbf{m}}}
\newcommand{\vn}{{\mathbf{n}}}
\newcommand{\vo}{{\mathbf{o}}}
\newcommand{\vp}{{\mathbf{p}}}
\newcommand{\vq}{{\mathbf{q}}}
\newcommand{\vr}{{\mathbf{r}}}
\newcommand{\vs}{{\mathbf{s}}}
\newcommand{\vt}{{\mathbf{t}}}
\newcommand{\vu}{{\mathbf{u}}}
\newcommand{\vv}{{\mathbf{v}}}
\newcommand{\vw}{{\mathbf{w}}}
\newcommand{\vx}{{\mathbf{x}}}
\newcommand{\vy}{{\mathbf{y}}}
\newcommand{\vz}{{\mathbf{z}}}
%
%\newcommand{\ta}{{\tilde{a}}}
%\newcommand{\tb}{{\tilde{b}}}
%\newcommand{\tc}{{\tilde{c}}}
%\newcommand{\td}{{\tilde{d}}}
%\newcommand{\te}{{\tilde{e}}}
%\newcommand{\tf}{{\tilde{f}}}
%\newcommand{\tg}{{\tilde{g}}}
%\newcommand{\th}{{\tilde{h}}}
%\newcommand{\ti}{{\tilde{i}}}
%\newcommand{\tj}{{\tilde{j}}}
%\newcommand{\tk}{{\tilde{k}}}
%\newcommand{\tl}{{\tilde{l}}}
%\newcommand{\tm}{{\tilde{m}}}
%\newcommand{\tn}{{\tilde{n}}}
%\newcommand{\to}{{\tilde{o}}}
%\newcommand{\tp}{{\tilde{p}}}
%\newcommand{\tq}{{\tilde{q}}}
%\newcommand{\tr}{{\tilde{r}}}
%\newcommand{\ts}{{\tilde{s}}}
%\newcommand{\tt}{{\tilde{t}}}
%\newcommand{\tu}{{\tilde{u}}}
\newcommand{\tv}{{\tilde{v}}}
\newcommand{\tw}{{\tilde{w}}}
%\newcommand{\tx}{{\tilde{x}}}
%\newcommand{\ty}{{\tilde{y}}}
\newcommand{\tz}{{\tilde{z}}}
\newcommand{\umu}{\overline{M}}
\newcommand{\lmu}{\underline{M}}

\newcommand{\vA}{{\mathbf{A}}}
\newcommand{\vB}{{\mathbf{B}}}
\newcommand{\vC}{{\mathbf{C}}}
\newcommand{\vD}{{\mathbf{D}}}
\newcommand{\vE}{{\mathbf{E}}}
\newcommand{\vF}{{\mathbf{F}}}
\newcommand{\vG}{{\mathbf{G}}}
\newcommand{\vH}{{\mathbf{H}}}
\newcommand{\vI}{{\mathbf{I}}}
\newcommand{\vJ}{{\mathbf{J}}}
\newcommand{\vK}{{\mathbf{K}}}
\newcommand{\vL}{{\mathbf{L}}}
\newcommand{\vM}{{\mathbf{M}}}
\newcommand{\vN}{{\mathbf{N}}}
\newcommand{\vO}{{\mathbf{O}}}
\newcommand{\vP}{{\mathbf{P}}}
\newcommand{\vQ}{{\mathbf{Q}}}
\newcommand{\vR}{{\mathbf{R}}}
\newcommand{\vS}{{\mathbf{S}}}
\newcommand{\vT}{{\mathbf{T}}}
\newcommand{\vU}{{\mathbf{U}}}
\newcommand{\vV}{{\mathbf{V}}}
\newcommand{\vW}{{\mathbf{W}}}
\newcommand{\vX}{{\mathbf{X}}}
\newcommand{\vY}{{\mathbf{Y}}}
\newcommand{\vZ}{{\mathbf{Z}}}

\newcommand{\cA}{{\mathcal{A}}}
\newcommand{\cB}{{\mathcal{B}}}
\newcommand{\cC}{{\mathcal{C}}}
\newcommand{\cD}{{\mathcal{D}}}
\newcommand{\cE}{{\mathcal{E}}}
\newcommand{\cF}{{\mathcal{F}}}
\newcommand{\cG}{{\mathcal{G}}}
\newcommand{\cH}{{\mathcal{H}}}
\newcommand{\cI}{{\mathcal{I}}}
\newcommand{\cJ}{{\mathcal{J}}}
\newcommand{\cK}{{\mathcal{K}}}
\newcommand{\cL}{{\mathcal{L}}}
\newcommand{\cM}{{\mathcal{M}}}
\newcommand{\cN}{{\mathcal{N}}}
\newcommand{\cO}{{\mathcal{O}}}
\newcommand{\cP}{{\mathcal{P}}}
\newcommand{\cQ}{{\mathcal{Q}}}
\newcommand{\cR}{{\mathcal{R}}}
\newcommand{\cS}{{\mathcal{S}}}
\newcommand{\cT}{{\mathcal{T}}}
\newcommand{\cU}{{\mathcal{U}}}
\newcommand{\cV}{{\mathcal{V}}}
\newcommand{\cW}{{\mathcal{W}}}
\newcommand{\cX}{{\mathcal{X}}}
\newcommand{\cY}{{\mathcal{Y}}}
\newcommand{\cZ}{{\mathcal{Z}}}

\newcommand{\ri}{{\mathrm{i}}}
\newcommand{\rr}{{\mathrm{r}}}


%% macros for math notions and operators

\newcommand{\FF}{\mathbb{F}}
\newcommand{\RR}{\mathbb{R}}
\newcommand{\NN}{\mathbb{N}}
\newcommand{\CC}{\mathbb{C}}
\newcommand{\HH}{\mathbb{H}}
\newcommand{\II}{\mathbb{I}}
\newcommand{\JJ}{\mathbb{J}}
\newcommand{\DD}{\mathbb{D}}
\newcommand{\GG}{\mathbb{G}}
\newcommand{\ZZ}{\mathbb{Z}}
\renewcommand{\SS}{{\mathbb{S}}}
\newcommand{\SSp}{\mathbb{S}_{+}}
\newcommand{\SSpp}{\mathbb{S}_{++}}
\newcommand{\sign}{\mathrm{sign}}
\newcommand{\vzero}{\mathbf{0}}
\newcommand{\vone}{{\mathbf{1}}}

%%Probability symbols.
\newcommand{\EE}{\mathbb{E}}
\newcommand{\mkT}{\mathfrak{T}}
\newcommand{\mkS}{\mathfrak{S}}
\newcommand{\mkQ}{\mathfrak{Q}}
\newcommand{\pr}{\mathrm{prod}}

\newcommand{\st}{{\text{s.t.}}} % subject to
\newcommand{\St}{{\mathrm{subject~to}}} % subject to
\newcommand{\op}{{\mathrm{op}}} % subscript for operator norm
\newcommand{\opt}{{\mathrm{opt}}} % subscript for optimal solution
%\newcommand{\supp}{{\mathrm{supp}}} % support
\newcommand{\Prob}{{\mathrm{Prob}}} % probability
\newcommand{\Diag}{{\mathrm{Diag}}} % vector -> diagonal matrix
%\newcommand{\diag}{{\mathrm{diag}}} % matrix diagonal -> vector
\newcommand{\dom}{{\mathrm{dom}}} % domain
\newcommand{\range}{{\mathrm{range}}} % domain
%\newcommand{\grad}{{\nabla}}    % gradient
\newcommand{\tr}{{\mathrm{tr}}} % trace
\newcommand{\TV}{{\mathrm{TV}}} % total variation
\newcommand{\Proj}{{\mathrm{Proj}}}
\newcommand{\prj}{{\mathbf{proj}}}
\newcommand{\prox}{\mathbf{prox}}
\newcommand{\reflh}{\refl^{\bH}}
\newcommand{\refl}{\mathbf{refl}}
\newcommand{\proxh}{\prox^{\bH}}
\newcommand{\minimize}{\text{minimize}}
\newcommand{\bgamma}{\boldsymbol{\gamma}}
\newcommand{\bsigma}{\boldsymbol{\sigma}}
\newcommand{\bomega}{\boldsymbol{\omega}}
\newcommand{\blambda}{\boldsymbol{\lambda}}
\newcommand{\bH}{\vH}
\newcommand{\bbH}{\mathbb{H}}
\newcommand{\bB}{\boldsymbol{\cB}}
\newcommand{\Tau}{\mathrm{T}}
\newcommand{\tnabla}{\widetilde{\nabla}}
\newcommand{\TS}{{\cT_{\mathrm{3S}}}}
\newcommand{\TFBS}{{\cT_{\mathrm{FBS}}}}
\newcommand{\TBFS}{{\cT_{\mathrm{BFS}}}}
\newcommand{\TDRS}{{\cT_{\mathrm{DRS}}}}
\newcommand{\TPRS}{{\cT_{\mathrm{PRS}}}}
\newcommand{\TFBFS}{{\cT_{\mathrm{FBFS}}}}
\newcommand{\TFDRS}{{\cT_{\mathrm{FDRS}}}}
\newcommand{\TVC}{{\cT_{\textnormal{CV}}}}
\newcommand{\best}{\mathrm{best}}
\newcommand{\kbest}{k_{\best}}
\newcommand{\diff}{\mathrm{diff}}
\newcommand{\xbar}{\bar{x}}
\newcommand{\xgbar}{\bar{x}_g}
\newcommand{\xfbar}{\bar{x}_f}
\newcommand{\xihat}{\hat{\xi}}
\newcommand{\xg}{x_g}
\newcommand{\xf}{x_f}
\newcommand{\du}{\mathrm{d}u}
\newcommand{\dy}{\mathrm{d}y}
\newcommand{\kconvergence}{\stackrel{k \rightarrow \infty}{\rightarrow }}
\newcommand{\Grph}{\mathrm{Grph}}
\DeclareMathOperator{\shrink}{shrink} % shrinkage
\DeclareMathOperator*{\argmin}{arg\,min}
\DeclareMathOperator*{\argmax}{arg\,max}
\DeclareMathOperator*{\Min}{minimize}
\DeclareMathOperator*{\Max}{maximize}
\DeclareMathOperator*{\Fix}{Fix}
\DeclareMathOperator*{\zer}{zer}    % the set of zeros of an operator
\DeclareMathOperator*{\nablah}{\nabla^{\bH}}
\DeclareMathOperator*{\gra}{gra}
\DeclarePairedDelimiter{\dotpb}{\langle}{\rangle_{\bH}}
\DeclarePairedDelimiter{\dotpv}{\langle}{\rangle_{\vH}}
\DeclarePairedDelimiter{\dotp}{\langle}{\rangle}
\DeclareMathOperator*{\as}{a.s.}
\newcommand{\nops}[2]{\ensuremath{\mathfrak{M}\left[{#1}\mapsto {#2}\right]}}


%% macros for environments math equations

\newcommand{\MyFigure}[1]{../fig/#1}

\newcommand{\bc}{\begin{center}}
\newcommand{\ec}{\end{center}}

\newcommand{\bdm}{\begin{displaymath}}
\newcommand{\edm}{\end{displaymath}}

\newcommand{\beq}{\begin{equation}}
\newcommand{\eeq}{\end{equation}}

\newcommand{\bfl}{\begin{flushleft}}
\newcommand{\efl}{\end{flushleft}}

\newcommand{\bt}{\begin{tabbing}}
\newcommand{\et}{\end{tabbing}}

\newcommand{\beqn}{\begin{align}}
\newcommand{\eeqn}{\end{align}}

\newcommand{\beqs}{\begin{align*}} % no equation numbers
\newcommand{\eeqs}{\end{align*}}  % no equation numbers

%
%\newtheorem{theorem}{Theorem}
%\newtheorem{thm}{Theorem}
%\newtheorem{assumption}{Assumption}
%%\newtheorem{condition}{Condition}
%%\newtheorem{rul}{Rule}
%\newtheorem{definition}{Definition}
%\newtheorem{corollary}{Corollary}
%\newtheorem{remark}{Remark}
%\newtheorem{lemma}{Lemma}
%\newtheorem{proposition}{Proposition}
%\newtheorem{example}{Example}
%\newtheorem{proof}{Proof}
%

%% macros for theorem-like environments
%\newtheorem{assumption}{Assumption}

%\newtheorem{theorem}{Theorem}
% \newtheorem{proof}{Proof}

%% AMSA theorem


%\newtheorem{example}[remark]{Example}


\usepackage{color}

\definecolor{myred}{rgb}{0.6,0,0}
\definecolor{mygreen}{rgb}{0,0.6,0}
\definecolor{mygray}{rgb}{0.5,0.5,0.5}
\definecolor{mymauve}{rgb}{0.58,0,0.82}

\newcommand\mycommfont[1]{\footnotesize\ttfamily\textcolor{myred}{#1}}
\SetCommentSty{mycommfont}

\lstset{ %
  columns=fullflexible,
  basicstyle=\ttfamily,
  backgroundcolor=\color{white},
  %backgroundcolor=\color{white},   % choose the background color; you must add \usepackage{color} or \usepackage{xcolor}
  % basicstyle=\footnotesize,        % the size of the fonts that are used for the code
  breakatwhitespace=false,         % sets if automatic breaks should only happen at whitespace
  breaklines=true,                 % sets automatic line breaking
  captionpos=b,                    % sets the caption-position to bottom
  commentstyle=\color{myred},    % comment style
  deletekeywords={...},            % if you want to delete keywords from the given language
  escapeinside={\%*}{*)},          % if you want to add LaTeX within your code
  extendedchars=true,              % lets you use non-ASCII characters; for 8-bits encodings only, does not work with UTF-8
  frame=single,	                   % adds a frame around the code
  keepspaces=true,                 % keeps spaces in text, useful for keeping indentation of code (possibly needs columns=flexible)
  keywordstyle=\color{blue},       % keyword style
  language=Octave,                 % the language of the code
  otherkeywords={*,...},           % if you want to add more keywords to the set
  numbers=left,                    % where to put the line-numbers; possible values are (none, left, right)
  numbersep=5pt,                   % how far the line-numbers are from the code
  numberstyle=\tiny\color{mygray}, % the style that is used for the line-numbers
  rulecolor=\color{black},         % if not set, the frame-color may be changed on line-breaks within not-black text (e.g. comments (green here))
  showspaces=false,                % show spaces everywhere adding particular underscores; it overrides 'showstringspaces'
  showstringspaces=false,          % underline spaces within strings only
  showtabs=false,                  % show tabs within strings adding particular underscores
  stepnumber=1,                    % the step between two line-numbers. If it's 1, each line will be numbered
  stringstyle=\color{mymauve},     % string literal style
  tabsize=2,	                   % sets default tabsize to 2 spaces
  title=\lstname,                  % show the filename of files included with \lstinputlisting; also try caption instead of title
  belowskip=0em,
  morekeywords={decltype, Vector},
}


\newcommand{\dataset}{{\cal D}}
\newcommand{\fracpartial}[2]{\frac{\partial #1}{\partial  #2}}

% Heading arguments are {volume}{year}{pages}{submitted}{published}{author-full-names}

\jmlrheading{1}{2016}{xx-xx}{x/xx}{xx/xx}{Brent Edmunds, Zhimin Peng and Wotao Yin}

% Short headings should be running head and authors last names

\ShortHeadings{ARock}{Brent Edmunds, Zhimin Peng and Wotao Yin}
\firstpageno{1}

\begin{document}

\title{ARock: An Asynchronous Parallel C++ Framework for Fixed Point Problems}

\author{\name Brent Edmunds \email  brent.edmunds@math.ucla.edu
       \AND
       \name Zhimin Peng \email zhimin.peng@math.ucla.edu
       \AND
	\name Wotao Yin \email wotaoyin@math.ucla.edu\\
       \addr Department of Mathematics\\
       University of California, Los Angeles\\
       Los Angeles, CA 90095, USA}	
\editor{xxx}

\maketitle

%% abstract
\begin{abstract}
ARock is an abstract framework for implementing sequential, sync-parallel, and asynchronous parallel algorithms on shared memory platforms. 
The programming model adopts the thread library from C++ at its lowest level. At the highest level, the goal is to enable fast prototyping of asynchronous parallel algorithms. 
We developed a multilevel approach to reduce the gap between expert to low-level programming and novice-level programming. 
A spectrum of applications have been implemented under this framework. 
Experiments demonstrate that ARock is very efficient on both dense and sparse datasets. 
\end{abstract}

\begin{keywords}
  Asynchronous Parallel, Operator Splitting, Optimization
\end{keywords}


% !TEX root = ./arock_pkg_main.tex
\section{Introduction}
Coordinate update (CU) methods reduce a large problem to smaller subproblems and are useful for solving large-sized problems.
These methods handle both linear and nonlinear maps, smooth and nonsmooth functions, and convex and nonconvex problems. Though CU methods have different settings and convergence properties,  the monotone operator theory can unify them into a single abstract framework. Specifically, CU method can be reduced to the coordinate update of a fixed-point iteration. 

Existing CU solvers \cite{,,,,} either focus on very particular problems or do not scale to large-sized problems due to their sequential implementation. Adapting them to solve slightly different large-sized problems requires significant human effort. 

In this paper, we present ARock, a C++ library to simplify the implementation of both sequential and parallel coordinate update algorithms. The library reduces the barrier to entry on doing async-parallel computing. Users of ARock can use the prebuilt solvers or build their own async-parallel solvers by knowing a few concepts of classic optimization formulations and algorithms.

ARock supports a wide range of prebuilt applications, including, but not limited to, linear equations, $\ell_1$ and $\ell_2$ regularized logistic regression, portfolio optimization, $\ell_1$ and $\ell_2$ regularized least squares, robust regression, quadratic programming, and nonnegative matrix factorization. The solvers of these applications can be interacted through easy-to-use command-line tools. Other features of ARock includes a rich set of operators, comprehensive documentation, and easy-to-use library calls.

Major details are
\begin{itemize}
\item list of current popular algorithms are coordinate based (sequential and parallel, asynchronous).
\item there is no framework for rapid prototyping of these algorithms, list existing code for doing coordinate update;
\item state fixed-point problem, steel some introduction from CF-paper;
\item We are releasing a unified framework for rapid prototyping of serial, sync-parallel and async-parallel coordinate algorithms;
\item It is simple enough that an undergraduate can implement it. 
\item Matlab interface is forth coming;
\end{itemize}


% !TEX root = ./arock_pkg_main.tex
\section{Quick Start}
\begin{itemize}
\item installation;
\item command line;
\item the app file;
\end{itemize}


% !TEX root = ./arock_pkg_main.tex
\section{Syntax}



%% !TEX root = ./arock_pkg_main.tex
\section{The Fixed-point Problem}
\begin{enumerate}
\item the fixed-point problem;
\item cached variables;
\item and refer everything else to CF-paper;
\end{enumerate}


%% !TEX root = ./arock_pkg_main.tex
\section{Related Works}
\begin{enumerate}
\item review existing for serial implementation and theory of CU method (steel some from the CF paper);
\item review existing sync-parallel solvers, so far we are aware of the followings 
\begin{itemize}
\item ac-dc package: \url{https://github.com/optml/ac-dc}, 
\end{itemize}
\item review existing async-parallel solver:
\begin{itemize}
\item Passcode: \url{http://www.cs.utexas.edu/~rofuyu/exp-codes/passcode-icml15-exp/}
\item Hogwild: \url{http://i.stanford.edu/hazy/victor/Hogwild/}
\end{itemize}

\end{enumerate}


%% !TEX root = ./arock_pkg_main.tex
\section{Organization}
\begin{enumerate}
\item Introduction
\item The fixed-point problem
\item The software package
	\begin{enumerate}
	\item Practical usage;
	\item Documentation;
	\item Design;
	\end{enumerate}
\item Comparison;

\item Conclusion 

\item Appendix (derivation of the examples??)

\end{enumerate}



% % !TEX root = ./arock_pkg_main.tex
\section{User Interaction}

\begin{itemize}
\item Implement apps (using library);
\item Using synchronous;
\item Implement your own operator;
\item implement your own coordinate update (put the entire update to a single operator)
\end{itemize}


% !TEX root = ./arock_pkg_main.tex
\section{Numerical Experiments}
In this section, we illustrate the efficiency of \pkg~for three applications: $\ell_1$ regularized logistic regression, portfolio optimization,  and nonnegative matrix factorization.


% !TEX root = ./arock_pkg_main.tex
\subsection{Minimizing Elastic Net Logistic Regression}
In this subsection, we apply ARock with FBS to the elastic net regularized logistic regression problem
\begin{equation}\label{eqn:l12_log}
\Min_{x \in \mathbb{R}^n} \lambda_1 \|x\|_1 + \frac{\lambda_2}{2} \|x\|^2_2 + \sum_{i=1}^N \log\big(1 + \exp(-b_i \cdot a_i^T x)\big),
\end{equation}
where $\{(a_i, b_i)\}_{i=1}^N$ is the set of sample-label pairs, $\lambda_1=0.001$, $\lambda_2 = 1.$, and $n$ and $N$ represent the numbers of features and samples, respectively. This test uses two libsvm datasets\footnote{\url{http://www.csie.ntu.edu.tw/~cjlin/libsvmtools/datasets/}}: news20, and url.

Figure \ref{fig:log_reg_obj} gives the running times of  the full update (sync-parallel) and ARock (async-parallel) implementations on the two datasets. Figure~\ref{fig:log_reg_speedup} is the speedup performance comparison of the two methods. We can observe that ARock achieves approximate-linear speedup, but sync-parallel scales poorly as we explain below. One can also see that ARock converges faster due to more relaxed forward operator step size selection.

\begin{figure}[!h]
        \centering
       \begin{subfigure}[b]{0.4\textwidth}
                \includegraphics[width=\textwidth]{./figs/news20_obj}
                \caption{news20}
        \end{subfigure}
        ~~
        \begin{subfigure}[b]{0.4\textwidth}
                \includegraphics[width=\textwidth]{./figs/url_obj}
                \caption{url}
        \end{subfigure} 
        \caption{Objective vs wall clock time.}\label{fig:log_reg_obj}
\end{figure}

\begin{figure}[!h]
        \centering
       \begin{subfigure}[b]{0.35\textwidth}
                \includegraphics[width=\textwidth]{./figs/news20_speedup}
                \caption{news20}
        \end{subfigure}
        ~~
        \begin{subfigure}[b]{0.35\textwidth}
                \includegraphics[width=\textwidth]{./figs/url_speedup}
                \caption{url}
        \end{subfigure}        
        \caption{Speedup vs number of threads.}\label{fig:log_reg_speedup}
\end{figure}

% !TEX root = ./arock_pkg_main.tex
\subsection{Portfolio Optimization}
% !TEX root = ./arock_pkg_main.tex
\subsection{Nonnegative matrix factorization}

We consider the following Nonnegative Matrix Factorization problem

\begin{equation*}
	\Min_{X \geq 0,Y \geq 0} ~\|A-X^T Y\|_2^2,
\end{equation*}
where $A \in \RR^{m\times n}$, $X \in \RR^{k \times m}$ and $Y \in \RR^{k \times m}$.
This problem, despite being nonconvex, has a special form.
The objective function is block multiconvex\footnote{the objective function is convex with respect to each variable when others are fixed} and its regularizers are separable. Problems of this type have been shown \citep{XuYin2013_block,XuYin2014_GloballyConvergent,BolteSabachTeboulle2014_proximal} to be amenable to coordinate update techniques and further amenable \citep{2016APALM} to the asynchronous regime.
As the problem is nonconvex, convergence is given to a local minimizer, not a global minimizer.

We run TMAC on a synthetic problem, $A=\hat X^T \hat Y$ ,  with $m=1000$ and $k=20$.
Elements of $\hat X$ and $\hat Y$ sampled independently from $N(0, 1)$ normal distribution, then threshold positive.
We ran the tests with variable number of threads and iterations.
The following results are the averages resulting from 20 runs.


\begin{figure}[!h]
        \centering
       \begin{subfigure}[b]{0.4\textwidth}
                \includegraphics[width=\textwidth]{./figs/NMF_plot}
                \caption{Objective vs walll clock time}
        \end{subfigure}
        ~~
        \begin{subfigure}[b]{0.4\textwidth}
                \includegraphics[width=\textwidth]{./figs/speedup}
                \caption{speedup}
        \end{subfigure}
        \caption{NMF results}\label{fig:NMF_speedup}
\end{figure}


To show scalability we increased the dimension of $k$, and tested the speedup.
\begin{table}[!h]
\centering
\begin{tabular}{rrrr}
 \toprule
{threads}&{k=10}&{k=20}&{k=100}\\\cmidrule{1-4}
1&1.00&1.00&1.00\\%\hline
2&1.97&1.98&1.98\\%\hline
4&3.75&3.75&3.76\\%\hline
8&7.12&7.33&7.35\\%\hline
16&13.38&14.51&14.43\\%\hline
\bottomrule
\end{tabular}
 \caption{\label{tab:nmf_res}Speedup results for nonnegative matrix factorization. }
\end{table} 
% !TEX root = ./arock_pkg_main.tex
\subsection{Minimizing Huber Loss}
Might delete this.
% !TEX root = ./arock_pkg_main.tex
\subsection{Minimizing Elastic Net Logistic Regression}
In this subsection, we apply ARock with FBS to the elastic net regularized logistic regression problem
\begin{equation}\label{eqn:l12_log}
\Min_{x \in \mathbb{R}^n} \lambda_1 \|x\|_1 + \frac{\lambda_2}{2} \|x\|^2_2 + \sum_{i=1}^N \log\big(1 + \exp(-b_i \cdot a_i^T x)\big),
\end{equation}
where $\{(a_i, b_i)\}_{i=1}^N$ is the set of sample-label pairs, $\lambda_1=0.001$, $\lambda_2 = 1.$, and $n$ and $N$ represent the numbers of features and samples, respectively. This test uses two libsvm datasets\footnote{\url{http://www.csie.ntu.edu.tw/~cjlin/libsvmtools/datasets/}}: news20, and url.

Figure \ref{fig:log_reg_obj} gives the running times of  the full update (sync-parallel) and ARock (async-parallel) implementations on the two datasets. Figure~\ref{fig:log_reg_speedup} is the speedup performance comparison of the two methods. We can observe that ARock achieves approximate-linear speedup, but sync-parallel scales poorly as we explain below. One can also see that ARock converges faster due to more relaxed forward operator step size selection.

\begin{figure}[!h]
        \centering
       \begin{subfigure}[b]{0.4\textwidth}
                \includegraphics[width=\textwidth]{./figs/news20_obj}
                \caption{news20}
        \end{subfigure}
        ~~
        \begin{subfigure}[b]{0.4\textwidth}
                \includegraphics[width=\textwidth]{./figs/url_obj}
                \caption{url}
        \end{subfigure} 
        \caption{Objective vs wall clock time.}\label{fig:log_reg_obj}
\end{figure}

\begin{figure}[!h]
        \centering
       \begin{subfigure}[b]{0.35\textwidth}
                \includegraphics[width=\textwidth]{./figs/news20_speedup}
                \caption{news20}
        \end{subfigure}
        ~~
        \begin{subfigure}[b]{0.35\textwidth}
                \includegraphics[width=\textwidth]{./figs/url_speedup}
                \caption{url}
        \end{subfigure}        
        \caption{Speedup vs number of threads.}\label{fig:log_reg_speedup}
\end{figure}

% !TEX root = ./arock_pkg_main.tex
\subsection{Portfolio Optimization}
% !TEX root = ./arock_pkg_main.tex
\subsection{Nonnegative matrix factorization}

We consider the following Nonnegative Matrix Factorization problem

\begin{equation*}
	\Min_{X \geq 0,Y \geq 0} ~\|A-X^T Y\|_2^2,
\end{equation*}
where $A \in \RR^{m\times n}$, $X \in \RR^{k \times m}$ and $Y \in \RR^{k \times m}$.
This problem, despite being nonconvex, has a special form.
The objective function is block multiconvex\footnote{the objective function is convex with respect to each variable when others are fixed} and its regularizers are separable. Problems of this type have been shown \citep{XuYin2013_block,XuYin2014_GloballyConvergent,BolteSabachTeboulle2014_proximal} to be amenable to coordinate update techniques and further amenable \citep{2016APALM} to the asynchronous regime.
As the problem is nonconvex, convergence is given to a local minimizer, not a global minimizer.

We run TMAC on a synthetic problem, $A=\hat X^T \hat Y$ ,  with $m=1000$ and $k=20$.
Elements of $\hat X$ and $\hat Y$ sampled independently from $N(0, 1)$ normal distribution, then threshold positive.
We ran the tests with variable number of threads and iterations.
The following results are the averages resulting from 20 runs.


\begin{figure}[!h]
        \centering
       \begin{subfigure}[b]{0.4\textwidth}
                \includegraphics[width=\textwidth]{./figs/NMF_plot}
                \caption{Objective vs walll clock time}
        \end{subfigure}
        ~~
        \begin{subfigure}[b]{0.4\textwidth}
                \includegraphics[width=\textwidth]{./figs/speedup}
                \caption{speedup}
        \end{subfigure}
        \caption{NMF results}\label{fig:NMF_speedup}
\end{figure}


To show scalability we increased the dimension of $k$, and tested the speedup.
\begin{table}[!h]
\centering
\begin{tabular}{rrrr}
 \toprule
{threads}&{k=10}&{k=20}&{k=100}\\\cmidrule{1-4}
1&1.00&1.00&1.00\\%\hline
2&1.97&1.98&1.98\\%\hline
4&3.75&3.75&3.76\\%\hline
8&7.12&7.33&7.35\\%\hline
16&13.38&14.51&14.43\\%\hline
\bottomrule
\end{tabular}
 \caption{\label{tab:nmf_res}Speedup results for nonnegative matrix factorization. }
\end{table} 

% !TEX root = ./arock_pkg_main.tex
\section{The Software Package}

% !TEX root = ./arock_pkg_main.tex
\subsection{Practical Usage}
\begin{itemize}
\item download;
\item script for downloading the required packages; 
\item script to do prepackage solvers;
\end{itemize}



% !TEX root = ./arock_pkg_main.tex
\subsection{Documentation}
\begin{itemize}
\item html docs;
\item heavily commented code; 
\item example code;
\item youtube HOWTO;
\end{itemize}


% !TEX root = ./arock_pkg_main.tex
\section{Architecture}


Writing efficient code is very different from writing an optimization algorithm.
Our toolbox's architecture is designed to mimic how a scientist writes down an optimization algorithm.
The toolbox achieves this by separating into the following layers: Numerical Linear Algebra, Operator, Scheme, Kernel, and Multicore Driver.
Each layer represents a different mathematical component of a parallel optimization algorithm.


The following is a brief description of each layer and how it interacts with the layers above and below it. \commzp{add the picture here}

\subsection{Numerical Linear Algebra}

We use Eigen, Sparse BLAS, and BLAS in our Toolbox. \commzp{add reference}
Directly using efficient numerical packages like BLAS can be intimidating due to complex function calls. We provide simplified function calls for common linear algebra operations like $a_i^T x$ in Algorithm \ref{alg:fbs_l1_log}. This layer insulates the user from the grit of raw numerical implementation. Higher layers use the Numerical Linear Algebra Layer in their implementations.


If our provided functions are not sufficient, documentation for eigen, Sparse BLAS, and BLAS can be found in the following locations. \commzp{documentation for these packages}

\subsection{Operator}

The Operator Layer contains Forward Operator objects (e.g., gradient descent step, subgradient step) and Backward Operator (e.g., proximal mapping, projection) objects. These operators types see heavy reuse throughtout optimization.
For instance, Algorithm \ref{alg:fbs_l1_log}, along with any other gradient based method for sparse logistic regression, requires the computation of the Forward Operator $x - \eta \, \nabla_x \,(\sum_{i = 1}^m \log (1 + \exp(-b_i \cdot a_i^T x)))$. On a similar vein, Nonnegative Matrix Factorization and Nonnegative Least Squares share a backwad operator, the projection onto the positive orthant.

Much as the Numerical Linear Algebra Layer insulates the user from the computation details of numerical linear algebra, the Operator Layer insulates the user from the computational details of operators. This is achieved by encapsulating the computation of common Forward and Backward Operators into Operator objects. \commwy{link back to code snippet, motivate readability, reusability, and fast coding} The Operator Layer, as a result, allows users to reason and code at the level of operators. The higher layers uses Operator objects as components in the creation of algorithms. \commzp{add footnote to explain the
differences between operator object and operator}

As our Toolbox is designed for coordinate update methods, each operator is implemented to compute coordinates efficiently. A coordinate or block of coordinates can be computed efficienty if the computational cost of a single coordinate or a block of coordinates of the operator is reduced by a dimensional factor compared to the evaluation of the entire operator (e.g. $(Ax)_i$ versus $Ax$). In some cases caching --- the storing an intermediate computation --- can improve the efficiency of updating a coordinate block. These ideas are formalized in CFU paper.

In the case that an operator is needed that we have not provided, a user can use CFU paper as a guide to identify coordinate-friendly structures and implement their own operator. In addition, there are certain rules about operator implementation that must be followed; see Section \ref{sc:implement}.


\subsection{Scheme}
A scheme describes how to make a single-iteration update to $x$.
It can be written as a combination of operators. For example, \eqref{eq:fbs_l1_log} is the Forward-Backward Splitting Scheme (also referred as Proximal Gradient Method) for a specific problem, sparse logistic regression. In Algorithm \ref{alg:fbs_l1_log}, it corresponds to Line 9. To apply the Forward-Backward Splitting Scheme to the sparse logistic regression problem \eqref{}, we need to specify a Forward Operator and a Backward Operator (e.g., on Line 5 of Algorithm \ref{alg:fbs_l1_log}). We can see this in code snippet above {\color{red} Need a way to reference the code snippet, can we embed it in a figure?}.
The scheme object \texttt{fbs} is specialized to sparse logistic regression by specifying its type as \texttt{ForwardBackwardSplitting<forward\_grad\_for\_log\_loss<SpMat>, prox\_l1>}.

We provide implemenations of the following schemes:  Proximal-Point Method,  Gradient Descent, Forward-Backward Splitting, Backward-Forward Splitting, Peaceman-Rachford Splitting, and Douglas-Rachford Splitting.

If the provided schemes are not sufficient, the user may implement their own scheme following certain rules so that their scheme can interact with the rest of the package; see Section \ref{sc:implement}.
The user is encouraged to use objects from the Operator Layer as building blocks, but direct calling the Linear Algebra Layer is perfectly functional. %Their are certain rules about implementing schemes that must be followed; see Section \ref{sc:implement}.

\subsection{Kernel}
A Kernel is a function that an agent executes.
%Agents, in our Toolbox, are realized as threads.
%C++11 threads are created from functions.
%The Kernel Layer contains the functions used to create agents.

As can be seen in Algorithm \ref{alg:fbs_l1_log}, an agent contains a coordinate choice rule and a scheme object.
The agent chooses a coordinate using its rule and call the Scheme object with the chosen coordinate to update $x$.
For each coordinate choice rule, we have implemented a corresponding Kernel function.

We provide the following coordinate choice rules: cyclic, random, and parallel Gauss-Seidel. The cyclic rule picks $i=(i_\mathrm{last} \mod n)+1$. \commwy{explain the rules.}
%block cyclic, and randomized block.
If the user desires a different coordinate rule, they may implement their own Kernel function, following the specifications in Section \ref{sc:implement}.


\subsection{Multicore driver}

%Consider algorithm \ref{alg:fbs_l1_log}.
Once the schemes and kernels have been specified, agents are responsible for carrying them out. Agents are realized as C++11 threads.
%However, agents must be created.
A Multicore Driver creates and manages such agents.
For example, if the user chooses the randomized block coordinate Kernel and ten agents, the Multicore Driver will create ten agents using that kernel. % using the user's specified scheme and the function in the Kernel layer corresponding to the randomized block coordinate rule.
The Multicore Driver is called with a Params object that contains parameters such as kernel choice, number of iterations, and step size.
An example of this is found in code snippet, where \texttt{fbs} and \texttt{params} are passed to the \texttt{MOTAC} driver.

Optionally, a Multicore Driver can launch a controller agent to control the other agents, for example, by choosing step sizes to accelerate convergence.
The current controller agent monitors convergence by periodically computing the fixed point residual.

Most users can treat this layer as a black box. Only when a new way to generate agents is desired, does the user need to modify this layer.

\section{Implementation Details}\label{sc:implement}

\subsection{Interaction between Layers}

The Operator, Scheme, and Kernel Layers interact heavily with each other. To formalize interaction between each layer, we introduce Layer Interfaces. A Layer Interface describes guaranteed member functions of objects in the Layer so that other Layers can safely use these member functions to interact. The Layer Interfaces allow for specialization of objects within each layer while still maintaining a uniform means of interaction. Consider the Operator Interface:
\begin{lstlisting}[language=C++,label={Operator_Interface}]
struct Operator_Interface {

  // returns the operator evaluated on v at the given index
  double operator() (Vector* v, int index);
  // returns the operator evaluated on val at the given index
  double operator() (double val, int index);
  // applies full operator to v_in and write it to v_out
  void operator() (Vector* v_in, Vector* v_out);
  // optional: see CFU paper
  void update_cache_vars (double old_x_i, double new_x_i, int index);
  // update the step size
  void update_step_size (double step_size_) ;
 };
\end{lstlisting}

For an object to belong to the Operator layer, it must have these functions defined.
Attempting to use an object as an Operator that does not have the functionality defined by the Operator Interface will result in compiler errors.

The Scheme Interface:
\begin{lstlisting}[language=C++]
struct Scheme_Interface {
  // update internal params
  void update_params(Params* params);
  // produce coordinate update associated with index
  double operator() (int index);
};
\end{lstlisting}
The Scheme Interface is very lightweight, as the update equations of optimization methods come in a variety of forms. All it requires is that a coordinate update can be produced and that used defined parameters can be passed to the object.
{{\color{red} need to add requirements derived from synchronous driver to Scheme Interface}


\subsection{Templating}

In C++, when objects have similar structures that only vary based upon an input type, templates are used to reduce code redundancy. For instance, the code for an object representing a dense matrix of doubles and the code for an object representing a dense matrix of floats is identical.
Templates are not objects, but instead are blueprints for constructing an object.
Based upon the arguments to the template, a corresponding type is automically constructed. see here *INSERT BRENT* for a primer on templates.
We use templating heavily in our Toolbox, as most of our workflow can be genericized.

Objects in the Scheme Layer are implemented as templates.
This is a natural choice, as object in the Scheme Layer map from optimization method's update to a specific optimization problem.
For example, in code snippet \ref{fbs_l1_log_code} the scheme type of \texttt{fbs} is defined by the arguments to the forward backward splitting template, \texttt{forward\_grad\_for\_log\_loss<SpMat>}, and \texttt{prox\_l1}.

Objects in the Object Layer are also templatized. Depending on data representation, it can be more efficient to use functionality designed for that data representation. Consider the difference between computing $x - \eta \, \nabla_x \,(\sum_{i = 1}^m \log (1 + \exp(-b_i \cdot a_i^T x)))$ when $a_i$ are sparse instead of dense.
Our linear algebra functions are overloaded so the compiler will deduce the proper function to use in the template.

Kernel functions are also templatized. This allows Kernel functions to take in as input arbitrary objects from the Scheme Layer. If we did not use templating, for each coordinate rule would need a function for every possible realization of gradient descent, proximal point method, etc.  Any object that satisfies the Scheme Interface can be passed to a Kernel function.

Multicore Drivers are templatized for the same reason as Kernel functions are templated. If we did not use templating, we would need a Multicore Driver for every possible realization of gradient descent, proximal point method, etc. Any object that satisfies the Scheme Interface can be passed to a Mutlicore Driver.

\subsection{How to write a kernel}
....

% !TEX root = ./arock_pkg_main.tex
\section{Future Work}

\pkg~ is still in development, and there are a variety of directions we plan to take the toolbox.

\subsection{Stochastic methods} 

Stochastic algorithms exploit summative structures in problems to produce cheap updates. Our current Toolbox does not currently support Stochastic algorithms, but can be modified to do so. Such a modification would require stochastic operators. A developement branch will appear shortly on our github after we release this paper.  

\subsection{Cluster computing}
 
 Currently, agents are realized as threads.
 This limits our toolbox to the single machine regime.
 Future releases intende to use MPI to bring \pkg~ to cluster computing. The new functionality will be provided in the Driver and the Kernel Layer. 

\subsection{User interface}
 \pkg~ requires the user to either use our prepackaged executables, or code moderately in C++.
 A graphic user interace is in development for those who wish to use only built-in operators and schemes.
 In addition, interfaces to Matlab and Python will be provided for algorithms of particular interest.
 
\subsection{Automatic parameter choice}
\pkg~ requires the user to choose stepsizes, and number of threads.
In future releases we intend to provide automatic stepsize heuristics. 
Optimizing thread number is a function of current processor usage and computing architecture. 
We intend to provide functions to survey the current architecture and suggest proper level of parallelism and asynchrony.

\subsection{Block coordinate update}
  Currently \pkg~ does not support block coordinate updates. Block coordinate updates present a tradeoff between iteration complexity and communication efficiency.
  Automatic block size deduction and block composition (the coordinates forming a block) is an open question.
We plan to explore several heuristics.

\subsection{New fields}

Splitting methods have been used in many fields (see "Some Facts about Operator-Splitting and
Alternating Direction Methods" for a more in depth discussion).
Our current release focuses on optimization, but provides a strong foundation for branching into other fields.
Fruitful fields to explore include:

\begin{itemize}
\item Time varying systems such as initial value problems and partial differential equations
\item Numerical Simulations
\item Large Scale Numerical Linear Algebra 
\end{itemize}


\section{Conclusion }
We have developed \pkg, an easy-to-use open source toolbox for large scale optimization problems.
The toolbox implemented both sequential and parallel algorithms based on operator splitting methods, stochastic methods,
and coordinate update methods. The architecture of \pkg~is separated into several layers and mimics how a scientist writes down an optimization algorithm. Therefore, it is easy for one to obtain a new algorithm by modifying just one of the layers such as adding a new operator.


\pkg~ is still being developed. New applications will be added to the package based upon new research and community input. The software and user guide \repo~ will be kept up-to-date and supported.


\bibliographystyle{siam}
\bibliography{asyn}

\end{document}
