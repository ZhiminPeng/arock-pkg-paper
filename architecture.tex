% !TEX root = ./arock_pkg_main.tex
\section{Architecture}

The toolbox can be broken into the following layers: Multicore Drivers, Kernels, Schemes, Operators, and Linear Algebra.
Each layer represents a different mathematical component. To maintain the mathematical abstraction between layers, Interfaces are used when necessary. 
Interfaces, a set of assume functionality of a layer, decouple a layer from the implementaiton details of the layers below it.
Consider the Interface for the Operator object:

\begin{lstlisting}[language=C++,label={Operator_Interface}]
struct Operator_Interface {
   // returns the operator evaluated on v at the given index
   double operator() (Vector* v, int index);
  // returns the operator evaluated on v at the given index
   double operator() (double val, int index);
  //applies full operator to v_in and write it to v_out
   void operator() (Vector* v_in, Vector* v_out);
  //optional: see CFU paper
  void update_cache_vars (double old_x_i, double new_x_i, int index);
  // update the step size
   void update_step_size (double step_size_) ;
 };
\end{lstlisting}

For an object to belong to the Operator layer, it must have these functions defined.
Attempting to use an object as an Operator that does not have the functionality defined by the Operator Interface will result in compiler errors.
Note that the Operator Intefarce only includes the function declarations, not the implementation of the functions themselves.

The following is a brief description of each layer and how it interacts with the layers above and below it.

\subsection{Multicore Drivers}

There is an implied step in algorithm \ref{alg:fbs_l1_log}: the creation of the agents running the coordinate update scheme.
The Multicore Drivers  layer is responsible for the creation of agents.
The psuedo code of algorithm \ref{alg:fbs_l1_log} provides a concrete recipe for creating an agent: a coordinate update scheme, a coordinate choice rule, a stepsize, and an iteration count.
At a high level, this section can be understood as a mapping from a coordinate update scheme to a multicore solver.

As seen in Section~\ref{sec:quick_start}, the \pkg~driver, takes as input, a scheme object, and a parameters object.
The parameter object determines how coordinates are chosen, how many iterations to run, and the stepsize.
Coordinate choices include cyclic, block cyclic, and randomized block. 
Based upon the parameter object, \pkg~launches creates  worker agents, agents that interact with the solver object to produce coordinate updates, and a controller agent, an agent that manages the worker threads.
Agents are realized as c++11 threads.

Most users can treat the Multicore Drivers as black box functionality.
As the details of a coordinate update scheme are packaged into a scheme object, our Multicore Drivers are insulated from most common code modficiations.
In the case that a different coordinate rule or a specialization of work distribution is desired, the Multicore Driver layer must be understood at a lower level.
To understand  Multicore Drivers later at a lower level, it is necessary to know how to launch and join threads.


\subsection{Kernel}


The Multicore Drivers Layer is responsible for creating agents, and the Kernel layer contains the agents types that can be created.
A worker agent chooses an index according to its rule, and, using a scheme object, computes a coordinate update.
As such, The Kernel layer can be seen as different realizations of the while loop stucture of algorithm \ref{alg:fbs_l1_log}.
In addition, the Kernel layer contains controller agents.
A controller agent manages the worker agents by choosing stepsizes to accelerate convergence.
The current controller agent monitors convergence by maintaining an approximate fixed point residual.

Agents are realized as c++11 threads and a c++11 thread is constructed from a function.
As a result, each worker and controller type, has  a corresponding function.
If the user wishes to introduce a new agent type, they need only write a function that implements the logic of the new agent.
They must then modify the Multicore Driver layer to recognize their new agent type as an option.

\subsection{Schemes}
Schemes are objects with, at minimum, member functions as defined by the Scheme Interface.

\begin{lstlisting}[language=C++]
struct Scheme_Interface {
void update_params(Params* params);
double operator() (int index);
};
\end{lstlisting}
By providing a coordinate to the scheme object, a coordinate update is applied.
 The Scheme Interface is very lightweight, as schemes can be quite varied in nature.
 We provide implemenations of the following schemes: \texttt{ForwardBackwardSplitting}, \texttt{ProximalPointAlgorithm},  \texttt{GradientDescent}, \texttt{BackwardForwardSplitting}, and \hfill \break      \texttt{PeacemanRachfordSplitting}.
Mathematically, Schemes are recipes for coordinate update.
To provide similar code functionality, schemes are templatized.  
For example, in \ref{fbs_l1_log_code} the type of \texttt{fbs} is defined by the arguments to the forward backward splitting template, \texttt{forward\_grad\_for\_log\_loss<SpMat>}, and \texttt{prox\_l1}. 
Different arguments to the template result in different versions of the scheme. 

\subsection{Operators}

Operators are objects with, at minimum, member functions as defined by \ref{Operator_Interface}, the Operator Interface.
The construction of an Operator object depends upon what is necessary to compute its coordinate update efficiently.
For example, \texttt{forward\_grad\_for\_log\_loss<SpMat>} requires the matrix \texttt{A}, the vector \texttt{b}, a stepsize \texttt{eta},  and, to maintain computational efficiency, the cached variable \texttt{Atx}.
In some cases the data may be dense, sparse, or best represented by a function (consider the wavelet tranform).
Operators, to allow for different data respresentations, are templatized on their data.
The matrix \texttt{A}, in this example, is sparse.




\subsection{Linear Algebra}

We use Eigen package to represent sparse matrices and vectors. 
Sparse linear algebra is provided through an interface to Eigen and Sparse BLAS.
Dense linear algebra is provided through an interface to BLAS. 







