% !TEX root = ./arock_pkg_main.tex
\section{Future Work}

\pkg~ is still in development, and there are a variety of directions we plan to take the toolbox.

\subsection{Stochastic methods} 

Stochastic algorithms exploit summative structures in problems to produce cheap updates. Our current Toolbox does not currently support Stochastic algorithms, but can be modified to do so. Such a modification would require stochastic operators. A developement branch will appear shortly on our github after we release this paper.  

\subsection{Cluster computing}
 
 Currently, agents are realized as threads.
 This limits our toolbox to the single machine regime.
 Future releases intende to use MPI to bring \pkg~ to cluster computing. The new functionality will be provided in the Driver and the Kernel Layer. 

\subsection{User interface}
 \pkg~ requires the user to either use our prepackaged executables, or code moderately in C++.
 A graphic user interace is in development for those who wish to use only built-in operators and schemes.
 In addition, interfaces to Matlab and Python will be provided for algorithms of particular interest.
 
\subsection{Automatic parameter choice}
\pkg~ requires the user to choose stepsizes, and number of threads.
In future releases we intend to provide automatic stepsize heuristics. 
Optimizing thread number is a function of current processor usage and computing architecture. 
We intend to provide functions to survey the current architecture and suggest proper level of parallelism and asynchrony.

\subsection{Block coordinate update}
  Currently \pkg~ does not support block coordinate updates. Block coordinate updates present a tradeoff between iteration complexity and communication efficiency.
  Automatic block size deduction and block composition (the coordinates forming a block) is an open question.
We plan to explore several heuristics.

\subsection{New fields}

Splitting methods have been used in many fields (see "Some Facts about Operator-Splitting and
Alternating Direction Methods" for a more in depth discussion).
Our current release focuses on optimization, but provides a strong foundation for branching into other fields.
Fruitful fields to explore include:

\begin{itemize}
\item Time varying systems such as initial value problems and partial differential equations
\item Numerical Simulations
\item Large Scale Numerical Linear Algebra 
\end{itemize}


\section{Conclusion }
We have developed \pkg, an easy-to-use open source toolbox for large scale optimization problems.
The toolbox implemented both sequential and parallel algorithms based on operator splitting methods, stochastic methods,
and coordinate update methods. The architecture of \pkg~is separated into several layers and mimics how a scientist writes down an optimization algorithm. Therefore, it is easy for one to obtain a new algorithm by modifying just one of the layers such as adding a new operator.


\pkg~ is still being developed. New applications will be added to the package based upon new research and community input. The software and user guide \repo~ will be kept up-to-date and supported.