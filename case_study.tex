% !TEX root = ./arock_pkg_main.tex
\section{Case Study}\label{sec:quick_start}
 To illustrate the practical usage of \pkg, we use the $\ell_1$ regularized logistic regression
\begin{equation}\label{eq:l1_log}
\min_{x \in \RR^n} \lambda \|x\|_1 + \sum_{i = 1}^m \log\left(1 + \exp(-b_i \cdot a_i^T x)\right)
\end{equation}
as an example, where $\{(a_i, b_i)\}_{i = 1}^m$ is the training dataset. We set the regularization parameter $\lambda$ to 1, and set the maximum number of epochs to 10. The following is the command to train the model on the news20\footnote{ This dataset is from \url{http://www.csie.ntu.edu.tw/~cjlin/libsvmtools/datasets}.} dataset with 2 threads,
\begin{lstlisting}[language=bash]
$ \pkg_fbs_l1_log -data news20.svm -epoch 10 -nthread 2 -lambda 1.
[other outputs skipped]
Computing time  is: 4.88(s).
\end{lstlisting}
where \texttt{-data, -epoch, -nthread, -lambda} are the flags for the data file, maximum number of epochs, total number of threads, and regularization parameter $\lambda$ respectively. We can see that the command-line tool is easy-to-use. Beyond the simplicity, \pkg is also efficient in the sense that the training time is less than 5 seconds for a problem with more than 1 million variables. Next, we show the major components of the source codes for building \texttt{\pkg\_fbs\_l1\_log}.

We solve \eqref{eq:l1_log} with the forward-backward splitting scheme
\begin{equation}\label{eq:fbs_l1_log}
x^{k+1} = \underbrace{\prox_{\lambda \eta \|x\|_1}}_{\text{backward operator}} (\underbrace{x^k - \eta \, \nabla \,\sum_{i = 1}^m \log (1 + \exp(-b_i \cdot a_i^T x^k)}_{\text{forward operator}}),
\end{equation}  
where the gradient step of logistic loss and the proximal operator of $\ell_1$ norm correspond to the forward step and backward step respectively. The following snippet of code (extracted from \texttt{apps/\pkg\_fbs\_l1\_log.cc}) implements \eqref{eq:fbs_l1_log} with the \pkg framework. Specifically, line 2 and line 5 define the forward operator and backward operator respectively. Line 3 and line 6 use \texttt{decltype} to simplify the names for the operator types. Line 7 define the FBS with the previously defined forward and backward operators. Line 8 calls \texttt{\pkg} on the \texttt{fbs}  object. We can see that creating an async-parallel solver can be easily achieved through assembling appropriate operators together.  
\begin{lstlisting}[language=C++]
  // [...] parameters are defined above
  // forward operator: gradient step for logistic loss
  forward_grad_for_log_loss<SpMat> forward(&A, &b, &Atx, eta);
  using F = decltype(forward); // yields the type of forward
  // backward operator: proximal operator for l1 norm 
  prox_l1 backward(eta, lambda);
  using B = decltype(backward); // yields the type of backward
  ForwardBackwardSplitting<F, B> fbs(&x, forward, backward);  
  \pkg(fbs, params);  
\end{lstlisting}

(Add the explanation for each line here. Show how easy it is to modify it to solve other problems. If you want know how the operators implemented, please refer to Section ???)
